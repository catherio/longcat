\documentclass{article}
\title{DS-GA-1008 Assignment 1}
\author{LongCat: Catherine Olsson and Long Sha and Kevin Brown}

\usepackage{graphicx}
\usepackage{caption}
\usepackage{subcaption}
\usepackage{amsmath}
\usepackage{enumerate}

\begin{document}
\maketitle

%Gaussian kernel: Didn't help. Above 0.35, maybe helped?
%Dimensionality of first two layers (nstates): To 128 (helped)
%Normalization kernel size: can only be run on CPU

\section{Submission details}

TODO what does our submission consist of

\section{Preprocessing}

We preprocessed each image in the following manner. First, we converted the
images from RGB representations to YUV, to separate luminance from color, with
the working assumption that pixels that are part of address numbers will have
different luminance, but possibly similar color. (Note: because torch's rgb2yuv
requires floating point representation, we recast the original representation of
the data. \\

Next, we normalized each feature globally, by subtracting the global mean for
each channel, and dividing by the standard deviation (zscore), for the train
data. We then used this mean and standard deviation to transform the test data
(to keep training and testing data separate). \\

CATHERINE ADD PREPROCESSING W/ GAUSSIAN KERNEL DETAILS:\\


\section{Architecture}

TODO summary paragraph. In particular:
\begin{itemize}
\item Number and type of layers
\item Number of neurons
\item Size of input
\end{itemize}

\section{Learning Techniques}
\begin{itemize}
\item Data augmentations?
\item Dropout?
\end{itemize}

\section{Training Procedure}
\begin{itemize}
\item Learning rate
\item Momentum
\item Error metrics
\item Train/validation split
\item Training/validation/test error
\end{itemize}

\end{document}
